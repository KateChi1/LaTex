\documentclass{beamer}		
\usetheme{Madrid}	
\usecolortheme{dolphin}	
\setbeamertemplate{navigation symbols}{}	
\date{}	
\title[]{Data Trends and Implications from Global Public Energy RD\&D Expenditures}	

\author[Myslikova et al. 2024]{Kate Hua-Ke Chi \inst{1}} 

\institute[] 
{
  \inst{1}
  The Fletcher School of Tufts University\\
  Climate Policy Lab
}

\date{December 2024}

\begin{document} 

\begin{frame}	
\titlepage	
\end{frame}		

\begin{frame}{To accelerate clean energy innovations}	

\begin{center}
    \includegraphics[height=4.5cm]{Screenshot.png}
\end{center}

\end{frame}

\begin{frame}{Global Public Energy RD\&D Expenditures Database}	

\begin{columns}	
\begin{column}{0.3\textwidth} 
\begin{text}
\paragraph{\small
- Set national objectives and priorities\\
- Determine market expectations\\
- Mobilize private sector\\
- Ensure human capital and knowledge flows; invest in supporting infrastructure\\}
\end{text}
\end{column}

\begin{column}{0.6\textwidth}
\begin{figure}
\includegraphics[height=5cm]{Picture1RDD.png}
\end{figure}
\end{column}
\end{columns}
\end{frame}

\begin{frame}
Global public energy RD\&D expenditures by technology type, 2000-2023

\begin{figure}
    \centering
    \includegraphics[width=0.8\linewidth]{Picture1.pdf}
    \caption{\label{} 
    \fontsize{6}{8} 
    \selectfont 
    The above plots the public expenditures by technology type from 2000 to 2023. Countries include Australia, Austria, Belgium, Brazil, Canada, Chile, China, Czech Republic, Denmark, Estonia, Finland, France, Germany, Greece, Hungary, India, Ireland, Italy, Japan, South Africa, South Korea, Lithuania, Luxembourg, Mexico, Netherlands, New Zealand, Norway, Poland, Portugal, Slovak Republic, Spain, Sweden, Switzerland, Republic of Türkiye, United Kingdom, United States, European Union. Data are taken from Myslikova et al. (2024). All numbers are expressed in USD (2023 prices and PPP). Numbers for Y2023 are largely incomplete.
}
\end{figure}
\end{frame}

\begin{frame}
\begin{figure}
    \centering
    \includegraphics[width=0.8\linewidth]{Top5.pdf}
    \caption{Top spenders in 2022: U.S., China, EU, France, and India 
}
    \label{fig:enter-label}
\end{figure}
\end{frame}

\begin{frame}
\begin{figure}
    \centering
    \includegraphics[width=0.75\linewidth]{USA.pdf}
    \caption{U.S. public energy RD\&D expenditures by technology type, 2000-2023
}
    \label{fig:enter-label}
\end{figure}
\end{frame}

\begin{frame}

\begin{figure}
 \centering
    \begin{minipage}{0.5\textwidth}
        \centering
        \includegraphics[width=1\textwidth]{Income.pdf} % first figure itself
        
    \end{minipage}\hfill

    \begin{minipage}{0.5\textwidth}
        \centering
        \includegraphics[width=1\textwidth]{Region.pdf} % second figure itself
       
    \end{minipage}
    \caption{Percentage share of public energy RD\&D expenditures by income group and region in 2022
}
\end{figure}
\end{frame}

\begin{frame}{Implications \& Acknowledgements}	
\begin{itemize} 
\item Public expenditures on the innovation of power and storage technologies are comparatively low and accounts for 6\% of total expenditures. \\
\item To increase energy access in developing countries → prioritize grid connectivity to remote areas and encourage development in an equitable manner. \\
\item Governments worldwide still allocated \$1.5 billion dollars in 2022 towards fossil fuels RD\&D.\\
\item This database began with data compilation of U.S. DOE RD\&D Spending by Kelly Sims Gallagher and Laura Diaz Anadon.\\
\item
https://www.climatepolicylab.org/rddmap\\

\end{itemize}
\end{frame}
\end{document}	